\documentclass[10pt,letterpaper]{article}
\pagestyle{myheadings}
\usepackage[utf8]{inputenc}
\usepackage[T1]{fontenc}            
\usepackage[spanish,es-tabla]{babel}
\usepackage{csquotes}
\usepackage{amsmath, amsfonts, amssymb}	
\usepackage{times}
\usepackage{multicol}
\usepackage{graphicx}
%\usepackage{hyperref}
\usepackage{url}
\usepackage{verbatim} 
\usepackage{lscape}
\usepackage[table]{xcolor}
\usepackage{tabularx,array,booktabs}
\newcolumntype{C}[1]{>{\centering\arraybackslash}p{#1}}
\newcolumntype{L}[1]{>{\raggedright\arraybackslash}p{#1}}
\usepackage{ltxtable}
\usepackage{caption} 
\captionsetup[table]{skip=10pt}
\usepackage[left=2cm,right=2cm,top=2cm,bottom=2cm]{geometry}
\usepackage{xcolor,listings}
\usepackage{textcomp}
\usepackage{tocloft} % Para centrar tableofcontent



%Color gris claro
\definecolor{Gray}{gray}{0.9}


\usepackage{listingsutf8}
\lstset{
inputencoding=utf8/latin1,
upquote=true,
  basicstyle=\small,
  breaklines=true,
  keywordstyle=\color{black}\bfseries\em,
  keywords={,input, output, return, datatype, function, in, if, else, foreach, while, begin, end, } %add the keywords you want, or load a language as Rubens explains in his comment above.
  }
\usepackage[numbib,nottoc,notlof,notlot]{tocbibind}
\addto\extrasspanish{\def\bibname{Referencias}\let\refname\bibname}

%para centrar indice
\renewcommand{\cfttoctitlefont}{\hspace*{\fill}\Huge\bfseries}
\renewcommand{\cftaftertoctitle}{\hspace*{\fill}}
\renewcommand{\cftlottitlefont}{\hspace*{\fill}\Huge\bfseries}
\renewcommand{\cftafterlottitle}{\hspace*{\fill}}
\renewcommand{\cftloftitlefont}{\hspace*{\fill}\Huge\bfseries}
\renewcommand{\cftafterloftitle}{\hspace*{\fill}}


\title{[Desarrollo de Software/Ingeniería de Software] Plantilla}

\begin{document}
\begin{titlepage}
		\begin{center}
        	\begin{center}
              \noindent% just to prevent indentation narrowing the line width for this line
              \includegraphics[height=0.1\textwidth]{usach.png}%
              \begin{minipage}[b]{0.7\textwidth}
              \centering
              {\textsc{Universidad de Santiago de Chile\\
              Facultad de Ciencia\\
              Departamento de Matemática y Ciencia de la Computación\\
              Licenciatura en Ciencia de la Computación}}
              \end{minipage}%
              \includegraphics[height=0.1\textwidth]{fac_cie.png}

            \end{center}

			\vspace{\fill}

			{Programación Avanzada}\\
			{Laboratorio N\textsuperscript{o} 1}
            %\vspace{1em} 
            \vspace{\fill} 
		\end{center}
        
        \begin{minipage}[l]{0.4\textwidth}
            \begin{flushleft}
            \linespread{1}
            \end{flushleft}
        \end{minipage}
        \begin{minipage}[l]{0.6\textwidth}
            \begin{flushright}
                \begin{flushleft}
                \large \textbf{\hspace{130pt}Integrantes: } 
                \end{flushleft}
                Fernando García\\Felipe Parra\\
                \begin{flushleft}
                \large \textbf{\hspace{130pt}Profesor: }
                \end{flushleft}
                Nicolas Theriault
                \begin{flushleft}
                \large \textbf{\hspace{130pt}Fecha de entrega: }
                \end{flushleft}
                23 de noviembre de 2018
            \end{flushright}
        \end{minipage}
	\end{titlepage}
	
    \clearpage 
    \setcounter{page}{0}
    \pagenumbering{roman}
    \setcounter{page}{1}
    \tableofcontents \thispagestyle{myheadings}
    \clearpage
    \setcounter{page}{0}
    \pagenumbering{arabic}
    \setcounter{page}{1}


\section{Introducción}

El objetivo principal de este trabajo es implementar algoritmos basicos de
aritmetica de polinomios: sumas, restas y multiplicación de dos polinomios.\\

Los objetivos secundarios son:
Implementar algoritmos clasicos de sumas, restas y multiplicacion de polinomios
de forma estable.\\

Analizar y aplicar un algoritmo del tipo dividir y conquistar, especificamente
el algoritmo de Karatsuba para la multiplicación de polinomios.\\
Obtener una implementación eficaz de la multiplicación de polinomios
aplicable para polinomios de cualquier grados, optimizada para la multiplicación de polinomios del mismo grado.

\section{Estructura de Datos Utilizadas}
Para los polinomios se creó un struct con nombre polinomio, el cual contiene las variables:\\\\
	- long* p;\\
	- long g;\\\\
long * p: es un arreglo el cual almacenará las constantes. Se eligió que fuera un arreglo ya que hace  más sencillo poder saber qué grado acompaña a cada constante.\\\\ 
Long g: es el grado que tendrá el polinomio. Es una buena opción ya que puede tener valores más grandes que un entero normal, lo mejor hubiera sido agregarle “const” ya que los grados nunca serían negativos y permitiría aún más espacio.


\section{Formato Utilizado}

los polinomios se iniciarán bajo la estructura\\\\ 
			$p(x) = a*x^i + b*x^{i+1} + c*x^{i+2} + ...  z*x^n$\\\\
las constantes son las que serán guardadas en cada espacio de la variable long *p. Esta estructura brinda que de manera automática, al buscar el elemento “i” del polinomio, el grado de la x que acompañará a esta constante será el valor de “i”. todos tendrán constantes y aquellos polinomios que no contengan ciertos grados de su x, serán representados como 0 T $x^{(gradoFaltante)}$.


\section{Estrategias Utilizadas}

Para la multiplicaciónn de polinomios con Reducir y Conquistar,
se soluciona el problema rebajando en un grado los polinomios
involucrados. \\Gracias a la propiedad distributiva de la multiplicación, se obtienen
dos nuevos polinomios, cada uno con un grado menor a los originales,
y se realiza la multiplicación clasica entre estos. \\Una vez realizada la suma y
obtenido el nuevo polinomio, este se multiplica por la x que se distribuyó con
anterioridad para conseguir el resultado.\\\\





En la multiplicacion inductiva con la estrategia de dividir �� y �� conquistar
se dividio el primer polinomio de grado n en dos polinomios: uno de grado n=2
o menor y otro que contenga los terminos d 
es posible de implementar con 4 multiplicaciones de polinomios de tama~no n=2
mas kn sumas (para algun k).\\\\


 
Para el metodo Karatsuba, se utiliza la misma estrategia utilizada en Dividir y
Conquistar, salvo que esta vez se crea una serie de polinomios auxiliares y se
aborda la multiplicacion como una suma de factores. El hecho de apoyarse en
las sumas para realizar la multiplicacion permite reducir el numero de multiplicaciones
(propiamente tales) a realizar.

\newpage
\section{Complejidad}

La complejidad de la suma es $O(m + n)$, con m y n grados de los respectivos polinomios, por lo que la complejidad en términos de n es $O(n)$\\\\

La complejidad de la resta es $O(m+n)$ con m y n grados de los respectivos polinomios, por lo que la complejidad en términos de n es O(n).\\\\

La complejidad de la multiplicación clasica es de la forma O(mn). Si los polinomios son del mismo grado, entonces es O($n^2$). Dado que tenemos polinomios con n términos ( de grado n -1 ), se obtendran $n^2$ productos de coeficientes y $n^2 - (2n-1) = (n-1)^2$ sumas de coeficientes. Con esto podemos concluir que el costo de memoria es de: $n^2 + (2n -1 )$ coeficientes. Para reducir la utilización de memoria se ocupa la estrategia reducir y conquistar en una de las entradas. De esta manera, se asociará el producto de polinomios de grado n-1 al producto de polinomios de grado n-2.
La recurrencia estará dado por la siguiente fórmula:\\

$C_{n} = C_{n-1} + ajustes$\\

Donde $C_{n}$, será el costo de multiplicar dos polinomios de grado n-1. Como resultado se obtendran $n^2$ multiplicaciones y $(n-1)^2$ sumas de coeficientes.

\section{Plataforma Computacional}

 Procesador: Intel i5 7300HQ\\
 Memoria RAM: 8 GB\\
 Sistema Operativo: Ubuntu 18.04\\
 Compilador: GCC 7.2.1\\
 Lenguaje: C11 \\

  \newpage
\section{Resultados Experimentales}

Multiplicación basica\\
\includegraphics[height=0.5\textwidth]{name/multiplicacionclasica.png}\\\\\\\\
Multiplicación reducir y conquistar
\\
\includegraphics[height=0.5\textwidth]{name/redyconqGrafico.png}\\
\newpage
Multiplicación dividir y conquistar
\\\\\\
\includegraphics[height=0.5\textwidth]{name/divyconqGrafico.png}\\
Multiplicacion Karatsuba\\\\\\
\includegraphics[height=0.5\textwidth]{name/karatsubagrafico.png}\\\\\
Con esto podemos comprobar que la multiplicación de Karatsuba es el método mas eficiente
para resolver problemas de multiplicación de polinomios de grados más grandes.
% \begin{thebibliography}{}
    %%%EJEMPLOS DE COMO TRABAJAR EN EL ENTORNO DE BIBLIOGRAFIA
    \bibitem{c1} 
    Madriz, G. (2018). Despliegue de la función de calidad DFC o la casa de la
    calidad. Recuperado de https://www.gestiopolis.com/despliegue-de-la-funcion-de-calidad-dfc-o-la-casa-de-la-calidad/
    
    \end{thebibliography}
 
%\section{Referencias}


\end{document}
