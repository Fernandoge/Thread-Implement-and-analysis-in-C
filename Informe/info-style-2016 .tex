%%%%%%%%%%%%%%%%%%%%%%%%%%%%%%%%%%%%%%%%%%%%%%%%%%%%%%%%%%%%%%%%%%%%%%%%%%%%%%
%
% Topico     : Estilo de Informes - DMCC  
% Autor      : Ruben Carvajal Schiaffino
% Santiago de Chile, 13/9/2016
%
%%%%%%%%%%%%%%%%%%%%%%%%%%%%%%%%%%%%%%%%%%%%%%%%%%%%%%%%%%%%%%%%%%%%%%%%%%%%%%
% 
%
\documentclass{report}
%
%
%
\usepackage{epsfig}
%
\usepackage{pdfpages}
%
\renewcommand*\thesection{\arabic{section}}
\newcommand \tab{\hspace*{25 pt}}
\newcommand \minitab{\hspace*{15 pt}}
%
\begin{document}
\begin{titlepage}
\begin{center}
%\psfig{figure=L-USACH-16.png,height=3cm,,}
\end{center}
\begin{center}
{\bf Departamento de Matem\'atica y Ciencia de la Computaci\'on}
\end{center}
\vspace{3cm}
\begin{center}
%%%%%%%%%%%%%%%%%%%%%%%%%%%%%%%%%%%%%%%%%%%%%%%%%%%%%%%%%%%%%%%%
%
% MODIFICAR. Despues del tag \bf se coloca el titulo del trabajo
%
{\Large \bf Laboratorio 1 \\
~~ \\
Algoritmo para Determinar si un Alfil Ataca a una Pieza Contraria}
%
%%%%%%%%%%%%%%%%%%%%%%%%%%%%%%%%%%%%%%%%%%%%%%%%%%%%%%%%%%%%%%%%
%
\end{center}
\begin{center}
%
%%%%%%%%%%%%%%%%%%%%%%%%%%%%%%%%%%%%%%%%%%%%%%%%%%%%%%%%%%%%%%%%
%
% MODIFICAR. Despues del tag \bf se coloca el semestre y año
%
{\large \bf Segundo Semestre 2016}
%
%%%%%%%%%%%%%%%%%%%%%%%%%%%%%%%%%%%%%%%%%%%%%%%%%%%%%%%%%%%%%%%%
%
\end{center}
\vspace{5cm}
\begin{tabular}{c l c}
%
%%%%%%%%%%%%%%%%%%%%%%%%%%%%%%%%%%%%%%%%%%%%%%%%%%%%%%%%%%%%%%%%
%
% MODIFICAR. En el primer campo colocar el nombre de la asignatura y su codigo
%            En el segundo campo colocar el nombre del autor
%
Programaci\'on Avanzada 26106 & ~~~~~~~~~~~~~~~~~ & Rub\'en Carvajal Schiaffino \\
%
%%%%%%%%%%%%%%%%%%%%%%%%%%%%%%%%%%%%%%%%%%%%%%%%%%%%%%%%%%%%%%%
%
% MODIFICAR. En el primer campo colocar el nombre de la carrera
%            En el segundo campo color direccion electronica
%
Licenciatura en Ciencia de la Computaci\'on & ~~ & ruben.carvajal@usach.cl 
%
%%%%%%%%%%%%%%%%%%%%%%%%%%%%%%%%%%%%%%%%%%%%%%%%%%%%%%%%%%%%%%%
%
\end{tabular}
\end{titlepage}
%
\section{Introducci\'on}
El objetivo de este trabajo es construir un algoritmo que dadas las coordenadas de dos piezas de ajedrez, de las 
cuales una es un alfil, determine si el alfil puede atacar a la otra pieza.
%
\section{Procedimiento}
Un alfil ataca a la pieza contraria si esta se encuentra en alguna de las diagonales 
Dado que cada pieza ocupa una posici\'on en el tablero es suficiente con calcular la pendiente de la recta que pasa 
por las coordenadas en las cuales se encuentran ambas piezas. 
\newline
\newline
Si el valor de dicha pendiente es $1$ o $-1$, entonces el alfil ataca a la pieza contraria.    
%
\subsection{Restricciones}
Se asume que las coordenadas de ambas piezas son v\'alidas y que cada ambas piezas no ocupan la misma posici\'on.
%
\section{Estructuras de Datos Utilizadas}
Para este problema no se ocupan estructuras de datos.
%
\section{Algoritmo}
El algoritmo para ...
\newline
\newline
{\bf Algorithm} Bootstrap
\newline
\newline
\begin{tabular}{l l}
{\bf Input}:  & $X, Y$ set of $n$ observations \\ 
              & $nbi$ Number of bootstrap iterations \\ 
              & $p$ percent of the confidence interval \\
{\bf Output}: & $x_l, x_r$ Confidence interval  \\
\end{tabular}
~~
\newline
\begin{tabular}{r l}
\\
1 & $\rho \leftarrow \mbox{\bf CorrelationCoef}(X,Y)$ \\
2 & {\bf for}~$i~\leftarrow 1~ \mbox{\bf to}~ nbi$~ {\bf do} \\
3 & \minitab $X',Y' \leftarrow \mbox{\bf SampleFrom}(X,Y,n)$ \\
4 & \minitab $cint_i \leftarrow \mbox{\bf CorrelationCoef}(X',Y')$ \\
5 & {\bf Sort}$(cint)$ \\
6 & $x_l,x_r \leftarrow \mbox{\bf ConfidenceInterval}(cint,p)$ \\
7 & $\mbox{\bf return}~x_l,x_r$
\end{tabular}
~~~
\newline
\newline
El algoritmo primero verifica si ambas piezas se encuentran en la misma fila (l\'inea 1) o en la misma columna
(l\'inea 3); en ambos casos el resultado es {\em Fail}. 
\newline
Si ambas piezas no se encuentran en la misma fila o columna se calcula la pendiente de la recta que pasa por las coordenadas
en las cuales se encuentran ambas piezas (l\'inea 5) y luego se verifica si el valor de la pendiente es $\pm 1$ 
(l\'inea 6).
%
\subsection{An\'alisis de Complejidad}
El algoritmo propuesto tiene costo constante.
%
\section{Implementaci\'on}
El algoritmo est\'a implementado en lenguaje C como se muestra en la figura a continuaci\'on:
%\begin{figure}[h]
%\includegraphics[scale=0.5]{alfil.pdf}
%\end{figure}
\newpage
%
\subsection{Plataforma Computacional}
El programa fue ejecutado eun un computador con las siguientes caracter\'isticas:
\begin{itemize}
\item {\bf Procesador:} Intel(R) Pentium(R) 4 CPU 3.00GHz
\item {\bf Memoria RAM:} 994312 kB
\item {\bf Sistema Operativo:} Linux - Ubuntu 9.04 
\end{itemize}
%
\section{Resultados Experimentales}
En este caso es trivial ...
%

\end{document}
